\chapter{はじめに}
\thispagestyle{myheadings}

% **DIOCOMO2019 ========================================================**
\section{背景および研究目的}

通信インフラの整備とIoTの発展により,現在では様々なモノがインターネットに繋がるようになった.
エアコンや照明,家の鍵から車に至るまで,数年前ではインターネットとは無関係だったものが今では当たり前のようにインターネットに接続されている.
IoTに対応した家電は遠隔からの操作や機器の状態の通知,使用ログの保存が可能であり,例えばIoTに対応したエアコンでは家の外から電源のON・OFFや電気使用量のログの確認ができる.
そうした利便性から日々新たなIoT家電が生み出され続けており,総務省が公開している資料\cite{soumusyo}によると2020年には約300億のIoTデバイスが稼働していると予想されている.
IoTデバイスの増加により家中にそれらが設置されると,そこから得られる使用データから行動パターンといったライフログのデータが取得できる.
ライフログのデータは,環境に合わせた電力制御や老人の異常行動の検知など幅広い分野への応用が期待できる.


\section{hogehoge}

\section{huga}
\subsection{hugahuga}
\subsubsection{hugahugahuga}


\begin{equation}
  \begin{split}
      &\mbox{活動量(kcal)} \\
      &= \mbox{METs} \times \mbox{体重(kg)} \times \mbox{時間(h)} \times 1.05 \label{CalFormula}
  \end{split}
\end{equation}

% \labelの後の文字列は参照の際に使う

しかし,これらのデータはIoTに対応した家電でないと取得できず,金庫や椅子,扉といった家電以外のモノではセンサが無いためデータ収集が不可能という問題がある.
この解決策として,(1)加速度センサや回転センサといったモーションセンサを使う方法や,(2)Wi-Fi電波のチャネル状態情報を用いる方法がある.
(1)の方法は直接モノの動きを観測できるため安定した精度で状態推定ができる反面,センサ単体での動作が難しく取得したデータの保存・解析に専用の機器が必要である.
また,(2)の方法では対象物のそれぞれにセンサを付ける必要が無い一方,状態の推定にWi-Fi電波の反射波を用いるという特徴から小さな対象物への適用は不向きである.

そこで,本研究では汎用的な機器のみで動作でき,かつ小さな対象物でも高い精度で状態推定できる手法を提案する.
具体的には日常の生活空間内における家電・家具の内部にBluetooth Low Energyビーコン(以降,BLEビーコンと呼称)を設置し,送信される電波をスマートフォンで受信した際の受信電波強度をもとにモノの状態推定を行う.
これまでBLEビーコンは広告配信や位置推定のための電波を発する文字通り「ビーコン」として使用されてきたが,物体内部に設置し「センサ」としての使い方をされている例はあまりない.
BLEビーコンから発信される電波は微弱であるため,障害物の有無により電波強度が変化しやすい.
そのため,モノの状態変化に合わせて受信電波強度にも変化が現れやすく,状態推定に適している.
また,モノにおける状態推定を行うとき,電池交換といった管理コストや機器の導入コストは少ない方が望ましい.
その点,BLEビーコンは省電力なため長期間の稼働が可能であり,サイズ・重さともに小さいため設置も容易である.
加えてBLEビーコンの電波はスマートフォンで受信できるため,電波の受信に特別な機器を必要とせず,UUID,major,minorの情報から各BLEビーコンの識別ができるため,1台のスマートフォンで複数のBLEビーコンの監視が可能である.

ただし,BLEビーコンの電波は微弱であるがゆえ,モノの状態変化に関係ないノイズの影響を受けてしまい,正常に推定を行えない可能性がある.
また,推定対象物の材質によっては状態が変化しても受信電波強度にあまり変化がでない場合も考えられる.
これら問題を解決するため,本研究では測定した電波強度情報へのデジタル処理とハードウェアの設置に工夫を行い,安定した状態推定を実現する手法を提案する.


% **DIOCOMO2020 (ryoga)====================================================**
\section{BLEビーコンを用いた状態推定の高齢者モニタリングへの応用}
\label{sec:example}

本研究では最終目標として,BLEビーコンを用いた状態推定の高齢者モニタリングへの応用も目指す.
日本では医療技術の発達や公衆衛生活動の発展より人の寿命が年々伸び続けており,厚生労働省が公開している『平成30年簡易生命表』\cite{AveLife}によると,2018年時点の平均寿命は男性が約81年,女性が約87年と公表されている.
この数字は1947年の平均寿命が男性は約50年,女性は約54年だという事実から考えると,71年間で男女ともに30年以上伸びている.


この平均寿命の延伸に伴う高齢化によって,要介護人や寝たきり状態の人も増加しており,ベッドや布団上での行動把握が介護人にとって重要になりつつある.
なぜならベッドや布団上での行動が把握できると,それを健康管理に繋げられるからである.
例えば,人は眠りの深さにより移動量が異なるため,寝返り回数などが把握できるとそこから睡眠の質を推定できる.
また要介護者がベッドや布団上で過ごす際,長時間同じ姿勢で寝ていると褥瘡発症の可能性が高まる.
同じ姿勢で寝ていると,一定の箇所に圧力がかかってしまうため血流の悪化や,汗や尿等による汚れにも繋がる.
そのため,介護人による定期的な体位の変更やシーツや衣服の取替,スキンケアなどが必要になる\cite{jokusou, mekanizumu}.
褥瘡を発症させないためには予防が重要となり,ベッド上での行動や睡眠位置の把握は様々なケアの指標になりうる.
褥瘡は介護を行う上で深刻な問題となっている.

褥瘡とは皮膚や皮膚の下部組織の損傷であり,長時間同じ姿勢で過ごし一定の箇所への圧力が継続した場合に発症する.
長期に渡り圧力がかかると血流が悪くなり,その結果皮膚組織に酸素や栄養が行き渡らなくなってしまい損傷してしまう.
一般的に褥瘡の重症度は6つのレベルに分類\cite{level}されており,軽い場合には皮膚が赤くなる程度だが,重症になると最終的には傷が筋肉や腱,骨にまで到達してしまう.
褥瘡の治療には除圧や最悪の場合手術が必要になり当人や介護人の負担が大きくなってしまうため,予防が重要となる.
そこで,1つ目の応用例としてBLEビーコンを用いた睡眠位置認識を提案する.

% **DIOCOMO2020 (ogane)====================================================**
また,平均寿命の延伸によって近年では人生100年時代と呼ばれるようになり,WHO(世界保健機関)はこれからは単に寿命を延ばすのではなく,健康寿命を延ばしていくのが大切だと提唱している.
健康寿命とはWHOが「健康上の問題で日常生活が制限されることなく生活できる期間」\cite{WHO}と定義したもので,この延長が実現できれば介護人の負担を軽減できるほか,医療費の削減に繋げられるだろうと期待されている.
健康寿命を延ばす重要な要素として,適度な運動やバランスの良い食事,質の高い睡眠などが挙げられるが特に運動に注目して考えてみると,運動量を知る指標として歩数を確認する方法が挙げられる.
歩数は単純に足を動かした回数であるため運動量の目標が立てやすく,その算出も容易である.

しかし,歩数は自立して歩ける人のみに適用できる指標であり,車椅子使用者ではこの指標を用いた運動量の推定は不可能である.
これまで車椅子における移動認識の先行研究として,GPSを用いた手法\cite{gps}や加速度,角速度,地磁気を用いた手法\cite{baske}など様々な手法が提案されてきた.
\cite{gps}の手法はGPS電波が受信できる屋外では高精度な位置推定が実現できる反面,GPS電波が届かない屋内では利用できないという問題がある.
また,\cite{baske}の手法はセンサを取り付けて直接車椅子の動きをセンシングするため,屋外・屋内のどちらでも位置推定ができる一方,定期的に電池交換といったメンテナンスが必要である.
本応用例は介護施設や老人ホームといった施設で,職員が車椅子を使用する入居者の健康管理を行うシーンを想定している.
そのため,低コストで導入でき,かつ電子機器の取り扱いに慣れていない人でも運用を行えるのが望ましい.
そこで,2つ目の応用例としてBLEビーコンを用いた車椅子の移動認識手法を提案する.

\section{論文構成}
本稿の構成は以下の通りである.
2章では,屋内日常物における状態推定に関する既存研究を紹介し,その問題点を述べる.
3章では,BLEビーコンを使用したモノの状態推定について述べ,その評価・考察を行う.
4章・5章では,BLEビーコンを用いた状態推定の応用例について述べる.
最後に6章で,本研究のまとめを行う.